The Packet Utilization Standard or PUS is an application-level interface standard for space-based applications. 
It is specified in reference [PS-SP]. 
In spite of its origin in the space industry, the PUS is suitable for a wide range of embedded control applications. 
In view of its long heritage and its proven ability to cover the interface needs of mission-critical systems of distributed applications, the PUS  has been used as a basis for the CORDET Framework in the sense that the service concept on which the CORDET Framework is based (see section \ref{sec:ServConcept}) is the same as the service concept specified by the PUS. 

In order to understand the degree of overlap between the PUS and the CORDET Framework, it is helpful to identify and contrast their respective concerns (the remainder of this section can be omitted by readers without a background in the space industry).

The PUS has two concerns: (a) it standardizes the semantics of the commands and reports which may be sent to or received from an application, and (b) it standardizes the external representations of these commands and reports (i.e. it specifies the layout of the packets which carry the commands and reports). 
The CORDET Framework shares the first concern in the sense that it uses the same service concept as the PUS but it does not share the second concern because it does not specify the external representation of commands and reports. 
Instead, the CORDET Framework specifies their internal representation (i.e. it predefines components to encapsulate commands and reports within an application) and it treats their serialization to, and de-serialization from, physical packets as an adaptation point to be resolved at application level.  
 
Thus, the CORDET Framework can be used to instantiate applications which are PUS-compliant but it is not restricted to PUS-compliant applications because it could be used to instantiate an application which uses a different external representation for its commands and reports than is specified by the PUS.

Table \ref{tab:PusCrConcerns} summarizes the concerns of the CORDET Framework and of the PUS.

\begin{longtable}{|>{\raggedright\arraybackslash}p{3cm}|p{10cm}|}
\caption{Concerns of CORDET Framework and of PUS}\label{tab:PusCrConcerns} \\
\hline
\rowcolor{light-gray}
\textbf{Concern} & \textbf{Coverage in CORDET Framework and PUS}\\
\hline\hline
\endfirsthead
\rowcolor{light-gray}
\textbf{Concern} & \textbf{Coverage in CORDET Framework and PUS}\\
\hline\hline
\endhead
Service Concept & CORDET Framework uses the same service concept as the PUS.\\
\hline
External Representation of Commands and Reports & The PUS specifies the external representation of its commands and reports (i.e. it specifies the layout of the packets carrying the commands and reports). The CORDET Framework does not specify the external representation of its commands and reports.\\
\hline
Internal Representation and Handling of Commands and Reports & The PUS does not specify how its commands and reports should be represented and handled inside an application. The CORDET Framework specifies the components representing the commands and reports in an application and the components required to handle them within that application.\\
\hline
\end{longtable}

In the CORDET Framework, a provider or user of services is called a \textit{CORDET application}. A CORDET application is identified by a single identifier. This identifier is used to designate the application in all its roles (source and/or receiver of commands and/or reports). 

In the PUS, the concept of \textit{application process} serves a similar purpose in the sense that commands and reports are sent and received by application processes (see section 5.4.2.1 of reference [PS-SP]). However, when an application process acts as source of reports and sink for commands (i.e. when it acts as a service provider), it is identified by an 11-bit integer called the \textit{Application Process Idengifier} or APID. When, instead, an application process acts as source of commands and sink for reports (i.e. when it acts as a service user), it is identified by a 16-bit integer called the \textit{Application Process User Identifier}. Thus, in the PUS world, the same physical application may be represented by different identifiers depending on its role. This means that mapping the identifier of a CORDET application to the identifiers of the corresponding PUS application processes is not straightforward.  

In order to allow such a mapping to be built, the CORDET Framework has introduced the concept of \textit{group}. In the CORDET world, the group is an attribute of commands and reports, which is implemented as a non-negative integer. The group identifier of a command/report is a function of that command/report's source, destination, type and sub-type. This function is, however, an adaptation point of the framework. This means that each application decides how to define the groups within an application. 

The mapping from the identifier of a CORDET application to the corresponding identifiers of the PUS application processes can then be done as follows:

\begin{itemize}
\item Each application process user identifier is mapped to a CORDET application identifier
\item Each application process identifier (APID) is mapped to a group identifier within a CORDET application 
\end{itemize}

Applications which do not need to achieve PUS compliance, can disregard the group concept and simply define one single group for all their commands and reports.